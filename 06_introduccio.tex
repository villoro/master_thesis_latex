\chapter{Introducció}

\section{Objectius del projecte}
L'objectiu d'aquest \ac{PFC} és fer un estudi de l'Experiència d'Usuari o \ac{UX} per una aplicació que serveixi per a gestionar les despeses personals domèstiques. Prèviament a l'estudi, es busca definir com es fan actualment els estudis \ac{UX} i quines parts tenen aquests tipus d'estudis. Després es procedirà amb l'estudi en si, on es busca definir com ha de ser una aplicació d'aquest tipus per a que garanteixi una bona \ac{UX}.

Concretament es busca dissenyar una aplicació que:
\begin{itemize}
\item Permeti enregistrar despeses i ingressos tot categoritzant-los.
\item Serveixi per a recordar els deutes (positius o negatius) que es tenen amb diverses persones.
\item Faciliti la gestió de despeses grupals alhora que permeti saldar els deutes minimitzant les transaccions entre els membres.
\item Sigui intuïtiva i senzilla de fer servir.
\item Visualment sigui agradable i minimalista per a que sigui agradable i còmode per a l'usuari.
\item Agradi als usuaris en general.
\end{itemize}

Per a dissenyar l'aplicació primer s'analitzarà com interaccionen els usuaris amb les aplicacions de gestió de despeses, per extreure'n les necessitats i les funcionalitats necessàries. Finalment es comprovarà la idoneïtat del disseny creat.

A més, també es busca resoldre els problemes d'enginyeria que es puguin derivar de les diverses funcions que haurà d'implementar l'aplicació. 

\section{Abast del projecte}
En aquest projecte s'estudiarà l'\ac{UX} per al tipus d'aplicacions esmentat tot definint quines utilitats i funcionalitats ha de tenir l'aplicació i com ha de ser la \ac{UI}. L'estudi s'enfocarà únicament en una aplicació per a \glspl{smartphone} que funcionen amb \gls{Android}, l'actual sistema operatiu més utilitzat\cite{Android_OS} en \glspl{smartphone} o telèfons intel·ligents. 
Finalment en quan al desenvolupament de l'aplicació, no es considera factible crear-la sencera amb totes els requeriments que es puguin deduir amb l'estudi de l'\ac{UX}. És per això que es faran proves de concepte, creant un o varis prototips que s'apropin el màxim possible a com hauria de ser l'aplicació. 