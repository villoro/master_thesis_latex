\chapter*{Llista d'acrònims}
\label{sec:glossary}
\addcontentsline{toc}{chapter}{Glossari}

%INFO: http://en.wikibooks.org/wiki/LaTeX/Glossary

\begin{acronym}
\acro{PFC}{Projecte Final de Carrera}
\acro{UX}{User Experience}
\acro{UI}{User Interface}
\acro{CSV}{Comma-Separated Values}
\acro{WAAD}{Work Activity Affinity Diagram}
\end{acronym}


\newglossaryentry{Android}
{
name = Android, description = Sistema operatiu mòbil amb el qual funcionen la majoria de telèfons mòbils \cite{Android_OS}
}

\newglossaryentry{conceptual_design}
{
name = disseny conceptual, plural = dissenys conceptuals, description = {és un tema, noció o idea amb el propòsit de comunicar una visió del disseny del sistema o producte. És la part del disseny del sistema que porta el model mental del dissenyador a la vida.}
}

\newglossaryentry{Google_play}
{
name = Google Play, description = Botiga virtual de Google en la qual es troben les aplicacions per a dispositius mòbils que funcionen amb \gls{Android} (\url{https://play.google.com/store/apps})
}

\newglossaryentry{Holo}
{
name = Holo, description = són unes directrius de disseny per a les aplicacions d'\gls{Android} que es van crear amb la versió 4.0
}

\newglossaryentry{Material}
{
name = Material Design, description = són unes directrius de disseny per a les aplicacions d'Android que es van crear amb la versió 5.0 per a substituir i millorar Holo. Per més informació \url{http://developer.android.com/design/material/index.html}
}

\newglossaryentry{workActivityNotes}
{
name = nota d'activitats de treball, plural = notes d'activitats de treball, description = es tracta de notes que parafrasegen i representen la opinió d'un usuari per a facilitar la comprensió de la opinió dels usuaris.  
}

%TODO Prototip vertical, horitzontal, T (p91)
%TODO UX

\newglossaryentry{smartphone}
{
name = \textit{smartphone}, description = Telefon intel·ligent que permeten realitzar tasques semblants a les realitzades per ordinadors 
}


%\newglossaryentry{persona}
%{
%name = personatge, description = és
%}

\printglossary


\chapter*{Apps}
\label{sec:apps}

\begin{tabular}{ | l | c | l | l | }
\hline
\textbf{Núm.} &  & \textbf{Nom} & \textbf{Autor} \\
\hline
App 1 & \includegraphics[scale=0.05]{A01_icon.png} & \href{https://play.google.com/store/apps/details?id=com.expensemanager}{Expense Manager} & Bishinews \\

App 2 & \includegraphics[scale=0.05]{A02_icon.png} & \href{https://play.google.com/store/apps/details?id=at.markushi.expensemanager}{Expense Manager} & Markus Hintersteiner \\

App 3 & \includegraphics[scale=0.05]{A03_icon.png} & \href{https://play.google.com/store/apps/details?id=com.bruno.myapps.droidwallet}{Droid Wallet} & William Bruno \\

App 4 & \includegraphics[scale=0.05]{A04_icon.png} & \href{https://play.google.com/store/apps/details?id=com.code44.finance}{Financius - Expense Manager} & Mantas Varnagiris \\

App 5 & \includegraphics[scale=0.05]{A05_icon.png} & \href{https://play.google.com/store/apps/details?id=com.handyapps.expenseiq}{Expense IQ} & Handy Apps \\

App 6 & \includegraphics[scale=0.05]{A06_icon.png} & \href{https://play.google.com/store/apps/details?id=com.techahead.ExpenseManager}{Diario Gasto Gerente (Daily Expense Manager)} & Gullak \\

App 7 & \includegraphics[scale=0.05]{A07_icon.png} & \href{https://play.google.com/store/apps/details?id=com.bookmark.money}{Money lover - Expense Manager} & ZooStudio   \\

App 8 & \includegraphics[scale=0.05]{A08_icon.png} & \href{https://play.google.com/store/apps/details?id=com.kpmoney.android}{AndroMoney (Expense Track)} & AndroMoney \\

App 9 & \includegraphics[scale=0.05]{A10_icon.png} & \href{https://play.google.com/store/apps/details?id=cz.destil.settleup}{Settle up} & David Vávra \\

App 10 & \includegraphics[scale=0.05]{A11_icon.png} & \href{https://play.google.com/store/apps/details?id=com.Splitwise.SplitwiseMobile}{Splitwise} & Splitwise \\

App 11 & \includegraphics[scale=0.2]{A12_icon.png} & Expensor & Arnau Villoro \\
\hline
\end{tabular}