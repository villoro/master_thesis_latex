\chapter*{Llista d'acrònims}
\label{sec:glossary}
\addcontentsline{toc}{chapter}{Glossari}

%INFO: http://en.wikibooks.org/wiki/LaTeX/Glossary

\begin{acronym}
\acro{PFC}{Projecte Final de Carrera}
\acro{UX}{\textit{User Experience}}
\acro{UI}{\textit{User Interface}}
\acro{CSV}{\textit{Comma-Separated Values}}
\acro{WAAD}{\textit{Work Activity Affinity Diagram}}
\end{acronym}


\newglossaryentry{Android}
{
name = Android, description = {Sistema operatiu mòbil amb el qual funcionen la majoria de telèfons mòbils \cite{Android_OS}}
}

\newglossaryentry{barrier}
{
name = barrera, description = {És un problema que interfereix amb les operacions que l'usuari executa normalment}
}

\newglossaryentry{BAAS}
{
name = \textit{Backend as a Service (BaaS}, description = {És un model per proveir als desenvolupadors d'apliacions mòbils una forma d'emmagatzematge de dades al núvol i altres funcions.}
}

\newglossaryentry{BB}
{
name = \textit{Branch and bound}, description = {És un algoritme d'optimització emprat, entre d'altres problemes, per trobar una solució amb variables enteres a partir d'una on totes les variables tenen valors reals.}
}

\newglossaryentry{affinity_diagram}
{
name = diagrama d'afinitat, description = {És una representació gràfica i jeràrquica que serveix per mostrar un resum extret de una gran quantitat d'informació quantitativa i qualitativa.}
}

\newglossaryentry{WAAD}
{
name = \textit{Work Activity Affinity Diagram}, description = {És un diagrama d'afinitat que serveix per filtrar i organitzar les notes d'activitat de treball extretes a l'anàlisi contextual. Serveix per mostrar les notes agrupades i organitzades segons similituds entre elles.}
}

\newglossaryentry{conceptual_design}
{
name = disseny conceptual, plural = dissenys conceptuals, description = {És un tema, noció o idea amb el propòsit de comunicar una visió del disseny del sistema o producte. És la part del disseny del sistema que porta el model mental del dissenyador a la vida.}
}

\newglossaryentry{User_Experience}
{
name = \textit{user experience} o experiència d'usuari, description = {És la totalitat de l'efecte o efectes que sent (o experimenta) internament l'usuari com a resultat de la interacció, i del context d'ús, amb el sistema, dispositiu o producte.}
}

\newglossaryentry{Google_play}
{
name = Google Play, description = {Botiga virtual de Google en la qual es troben les aplicacions per a dispositius mòbils que funcionen amb \gls{Android} (\url{https://play.google.com/store/apps})}
}

\newglossaryentry{Holo}
{
name = Holo, description = {Són unes directrius de disseny per a les aplicacions d'\gls{Android} que es van crear amb la versió 4.0.}
}

\newglossaryentry{contextual_inquiry}
{
name = \textit{investigació contextual}, description = {És una activitat que forma part d'un estudi d'UX que consisteix en recopilar informació detallada sobre com els usuaris interaccionen amb un producte/sistema. La seva finalitat és poder construir i/o millorar el producte/sistema i és du a terme mitjançant entrevistes i observacions dels usuaris.}
}

\newglossaryentry{inventari_tasques}
{
name = inventari de tasques jeràrquic, description = {És un model d'ús que representa les diferents tasques que es poden executar de forma jeràrquica. Dins l'estructura un fill és una subtasca del seu pare, no una tasca posterior en el temps.}
}

\newglossaryentry{lp_solve}
{
name = LP solve, description = {Llibreria per a resoldre problemes de programació lineal, incloent PLEM [\url{http://lpsolve.sourceforge.net/}]}
}

\newglossaryentry{Material}
{
name = Material Design, description = {Són unes directrius de disseny per a les aplicacions d'Android que es van crear amb la versió 5.0 per a substituir i millorar Holo. Per més informació \url{http://developer.android.com/design/material/index.html}}
}

\newglossaryentry{flow_model}
{
name = model de flux, description = {És una representació gràfica que explica com les diferents entitats es comuniquen per tal d'aconseguir que el treball es realitzi.}
}

\newglossaryentry{models_interaccio_tasques}
{
name = models d'interacció de les tasques, description = {És un model d'ús que representa l'execució pas a pas d'una tasca, incloent l'objectiu de la tasca, catalitzadors i les accions dels usuaris.}
}

\newglossaryentry{mockup}
{
name = \textit{mockup}, description = {És una representació estàtica de fidelitat mitja o alta.  Representa l'estructura de la informació i demostra les funcionalitats bàsiques d'una manera estàtica.}
}

\newglossaryentry{Nexus_5}
{
name = Nexus 5, description = {\textit{Smartphone} creat expressament per a desenvolupadors al 2013.}  
}

\newglossaryentry{workActivityNotes}
{
name = nota d'activitats de treball, plural = notes d'activitats de treball, description = {Es tracta de notes que parafrasegen i representen la opinió d'un usuari per a facilitar la comprensió de la opinió dels usuaris.}  
}

\newglossaryentry{work_role}
{
name = rol de treball, plural = rols de treball, description = {Corresponen als deures, funcions i activitats que desenvolupa una persona amb cert lloc de treball.}  
}

\newglossaryentry{persona}
{
name = personatge, description = {És una representació utilitzada en el disseny per a representar una persona fictícia amb un rol de treball concret. Serveix per a ajudar als dissenyadors a imaginar-se millor per a qui estan dissenyant}  
}

\newglossaryentry{prototip}
{
name = prototip, description = {És una representació navegable del producte final i normalment és de fidelitat mitja o alta. Simula la interacció amb la interfície d'usuari i permet que l'usuari experimenti interactuant amb la interfície i el contingut. També permet que provi les principals interaccions d'una manera molt similar al producte final.}
}

\newglossaryentry{prototip_h}
{
name = prototip horitzontal, plural = prototips horitzontals, description = {És un prototip què té moltes funcions però amb una profunditat baixa en quan a funcionalitat.}
}

\newglossaryentry{prototip_l}
{
name = prototip local, plural = prototips locals, description = {És un prototip que només implementa unes poques funcions i amb poca profunditat en quan a funcionalitat.}
}

\newglossaryentry{prototip_v}
{
name = prototip vertical, plural = prototips verticals, description = {És un prototip amb una profunditat alta de funcionalitat però que es centra només en poques funcions}
}

\newglossaryentry{sketch}
{
name = \textit{sketch}, description = {És un dibuix ràpid que es realitza a mà alçada que reprodueix un concepte, idea o la generalitat d'un projecte de manera molt senzilla. És una representació de baixa fidelitat} 
}

\newglossaryentry{smartphone}
{
name = \textit{smartphone}, description = {Telefon intel·ligent que permeten realitzar tasques semblants a les realitzades per ordinadors} 
}

\newglossaryentry{SQLite}
{
name = \textit{SQLite}, description = {Llenguatge per a bases de dades relacionals basat en SQL.} 
}

\newglossaryentry{SUS}
{
name = \textit{SUS}, description = {Qüestionari d'UX amb 10 preguntes que permet puntuar un producte o sistema} 
}

\newglossaryentry{work}
{
name = treballar, description = {Dins de l'àmbit d'UX, és interactuar o utilitzar el sistema/producte.} 
}

\newglossaryentry{wireframe}
{
name = \textit{wireframe}, description = {És una representació estàtica de baixa fidelitat d'un disseny. Mostra la estructura general així com les diferents parts que la componen representat per caixes o formes}
}




\printglossary


\chapter*{Apps}
\label{sec:apps}

\begin{tabular}{ | l | c | l | l | }
\hline
\textbf{Núm.} & \textbf{Icona} & \textbf{Nom} & \textbf{Autor} \\
\hline
App 1 & \includegraphics[scale=0.05]{A01_icon.png} & \href{https://play.google.com/store/apps/details?id=com.expensemanager}{Expense Manager} & Bishinews \\

App 2 & \includegraphics[scale=0.05]{A02_icon.png} & \href{https://play.google.com/store/apps/details?id=at.markushi.expensemanager}{Expense Manager} & Markus Hintersteiner \\

App 3 & \includegraphics[scale=0.05]{A03_icon.png} & \href{https://play.google.com/store/apps/details?id=com.bruno.myapps.droidwallet}{Droid Wallet} & William Bruno \\

App 4 & \includegraphics[scale=0.05]{A04_icon.png} & \href{https://play.google.com/store/apps/details?id=com.code44.finance}{Financius - Expense Manager} & Mantas Varnagiris \\

App 5 & \includegraphics[scale=0.05]{A05_icon.png} & \href{https://play.google.com/store/apps/details?id=com.handyapps.expenseiq}{Expense IQ} & Handy Apps \\

App 6 & \includegraphics[scale=0.05]{A06_icon.png} & \href{https://play.google.com/store/apps/details?id=com.techahead.ExpenseManager}{Diario Gasto Gerente (Daily Expense Manager)} & Gullak \\

App 7 & \includegraphics[scale=0.05]{A07_icon.png} & \href{https://play.google.com/store/apps/details?id=com.bookmark.money}{Money lover - Expense Manager} & ZooStudio   \\

App 8 & \includegraphics[scale=0.05]{A08_icon.png} & \href{https://play.google.com/store/apps/details?id=com.kpmoney.android}{AndroMoney (Expense Track)} & AndroMoney \\

App 9 & \includegraphics[scale=0.05]{A10_icon.png} & \href{https://play.google.com/store/apps/details?id=cz.destil.settleup}{Settle up} & David Vávra \\

App 10 & \includegraphics[scale=0.05]{A11_icon.png} & \href{https://play.google.com/store/apps/details?id=com.Splitwise.SplitwiseMobile}{Splitwise} & Splitwise \\

App 11 & \includegraphics[scale=0.2]{A12_icon.png} & Expensor & Arnau Villoro \\
\hline
\end{tabular}