\chapter{Conclusions}

Al llarg d'aquest \ac{PFC} s'ha fet un estudi d'\ac{UX} enfocat a una aplicació per a la gestió de despeses personals. Pel que fa als estudis \ac{UX} en si, s'ha deduït:

\begin{itemize}
\item L'anàlisi de com interaccionen els usuaris, sobretot a la primera iteració, és molt útil per entendre que volen els usuaris alhora que permet familiaritzar-se amb el producte/sistema a dissenyar.
\item Pel que fa al disseny i prototipatge és important començar amb una fidelitat i anar incrementant-la a mesura que s'itera. D'aquesta manera s'estalvien molts recursos a les primeres iteracions, i és que no és útil detallar un disseny quan aquest és susceptible de ser canviat. %TODO Flinto al glossari
\item Amb les aplicacions com Flinto (www.flinto.com) es poden animar molt ràpidament dissenys per a crear prototips, però aquests no tindran la flexibilitat dels prototips de paper. En productes/sistemes simples, com l'estudiat, es creu que és suficient amb els prototips creats amb Flinto o semblants, però si el producte/sistema és complex seria millor emprar prototips de paper.
\item Tot i ser un procés iteratiu, si a la primera iteració s'obtenen molt bons resultats s'estalvien molts recursos al llarg de l'estudi. 
\item Pocs usuaris estan disposats a efectuar entrevistes o enquestes llargues. Quants menys productes (dels que ja existeixin) hagin d'avaluar, més usuaris estaran disposats a avaluar-los. 
\item Avaluar 10 aplicacions és pesat per als usuaris, si s'hagués treballat amb 4 o 5 hauria estat millor. Tot i això, al partir l'avaluació sumarial en 2 parts, fent que a la segona només s'avaluessin 3 aplicacions a facilitat obtenir dades de molts usuaris diferents. 
\item A l'avaluació sumarial, uns qüestionaris donen una bona idea de la qualitat obtinguda. Tot i això, si és possible per recursos, és recomanable complementar-ho obtenint mètriques \ac{UX}.
\item Tot i que existeixen moltíssimes aplicacions per gestionar les despeses personals, hi ha poques que garanteixin una bona \ac{UX}. I, evidentment, fer un estudi d'\ac{UX} garanteix que l'aplicació resultat agradi als usuaris. 
\item Un estudi d'\ac{UX} pot ser molt complex i requerir la participació d'un equip molt nombrós. És per això que és important adaptar-lo als recursos disponibles, tant econòmics, com personals o de temps. 
\end{itemize}

Pel que fa a les aplicacions, tant en general com les que serveixen per gestionar despeses, s'ha arribat a la conclusió que:

\begin{itemize}
\item Les parts que impliquen més feina al programar sovint són les que els usuaris els hi paren menys atenció, ja que és habitual que les considerin òbvies. En aquest cas el repartiment de les despeses grupals ha estat una de les parts més complicades i pocs usuaris li han parat atenció.
\item Hi ha moltes funcions que no emocionen positivament als usuaris, però que en cas que no funcionin correctament sí que tenen un fort impacte negatiu en la seva \ac{UX}. La navegació per l'aplicació n'és un exemple, quan es pot navegar correctament l'usuari ni se n'adona, però en cas que no ho pugui fer es frustrarà.
\item L'apartat visual és molt important pels usuaris, si una aplicació no és maca, és complicat que agradi. Per tant és molt important que qualsevol prototip que vegin els usuaris tingui un bon acabat visual. 
\end{itemize}

I respecte a l'aplicació/prototip creat:

\begin{itemize}
\item S'ha aconseguit crear un prototip bastant semblant a com hauria de ser el producte final.
\item Si bé és cert que no s'ha pogut implementar totes les funcions que havia de tenir l'aplicació, si que comptava amb les funcions més importants, donant una bona idea de com serà l'aplicació si es publica en un futur. (Veure secció \ref{objectius})
\item L'aplicació compleix les funcions esmentades a l'apartat d'objectius encara que li manquin varies funcions necessàries segons l'anàlisi dels usuaris.
\item El prototip final ha excedit les expectatives si es té en compte que només es creia factible fer proves de concepte.
\item El prototip té alguns errors importants d'\ac{UX} dels quals se'n té constància però que no s'han corregit per manca de temps o coneixements a l'hora de programar. (Veure annex 4)
\end{itemize}

\section{Futur de l'estudi/aplicació}
L'estudi d'\ac{UX} d'aquest projecte s'ha efectuat per a dissenyar com ha de ser una aplicació per la gestió de l'economia personal. S'ha dissenyat amb la intenció de que en un futur es pugui implementar l'aplicació completament i penjar-la a les botigues d'aplicacions per \gls{Android}, encara que aquesta última part estigués fora de l'abast d'aquest projecte. 

Per a arribar a publicar l'aplicació a les botigues, caldrà que:

\begin{enumerate}
\item S'arreglin els problemes d'\ac{UX} i de programació en si de l'aplicació. (Veure annex 4.1 i 4.2)
\item S'implementin les funcionalitats descrites al disseny i que no implementa el prototip final. (Veure annex 4.3)
\item Crear una versió Beta del producte per a que la provin alguns usuaris a canvi de la seva opinió i que reportin possibles problemes que no s'hagin descobert.
\item Provar l'aplicació en diversos dispositius, comprovant que s'adeqüi correctament a diverses mides de pantalla i versions d'\gls{Android} diferents.
\item Dissenyar i desenvolupar versions especials per a \textit{Tablets} i altres aparells amb pantalla gran. 
\end{enumerate}