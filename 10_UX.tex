
\chapter{Experiència d'Usuari}
\section{Que s'entén per Experiència d'Usuari}
Segons Rex Hartson (2012, p. 19)\cite{UX_Book}, l'\ac{UX} és la totalitat de l'efecte o efectes que sent (o experimenta) internament l'usuari com a resultat de la interacció, i del context d'ús, amb el sistema, dispositiu o producte. És a dir, una bona \ac{UX} es produirà quan l'usuari gaudeixi interaccionant i utilitzant el dispositiu o producte. Interacció i ús s'empren en un sentit molt ampli, ja que inclouen veure, tocar, pensar sobre el producte/dispositiu o fins i tot admirar-lo. 
A més, l'\ac{UX} també engloba la usabilitat i la utilitat. Certament l'usuari sent internament parts de la usabilitat, com l'augment de productivitat. Però hi ha certes manifestacions de usabilitat, com podria ser el temps invertit en la tasca, que representa un component no necessàriament experimentat internament per l'usuari.

\section{Com s'estudia l'Experiència d'Usuari?}
L'\ac{UX} no pot ser dissenyada ja que depèn, no només del producte en si mateix, sinó que també depèn de l'usuari i la situació en la que l'utilitza [Smashing Magazine, 2012, p. 25-28]\cite{Smashing_User_Experience_Design}. I és que no és possible dissenyar ni l'usuari ni la situació. Però el que si es pot fer és dissenyar per a una bona \ac{UX}. Seguint els passos que proposa Rex Harton en el seu llibre d'\ac{UX} \cite{UX_Book}, per a aconseguir-ho hi ha quatre activitats elementals que són anàlisi, disseny, implementació i avaluació, tal i com es pot veure a la figura \ref{fig:UX_lifecycle}. Per tal d'aconseguir proporcionar una bona \ac{UX} aquestes activitats és duen a terme de forma iterativa, ja que no sempre és possible trobar un bon disseny al primer intent.

\begin{figure}[htp]
\centering
\includegraphics[scale=0.6]{UX_wheel.png}
\caption{Activitats a seguir per a dissenyar garantint una bona \ac{UX}}\label{fig:UX_lifecycle}
\end{figure}

Aquestes quatre activitats, a grans trets, consisteixen en:
\begin{description}
\item [Anàlisi] Es basa en entendre les necessitats de l'usuari que utilitzarà el producte.
\item [Disseny] Consisteix en la creació de dissenys conceptuals determinant la interacció, el comportament i l'aparença del producte.
\item [Implementació] Correspon a la creació del prototip.
\item [Avaluació] Es tracta de comprovar si el disseny satisfà les necessitats dels usuaris que s'han determinat.
\end{description}

\subsection{Anàlisi}\label{sec:analisi}
L'objectiu general d'aquesta activitat és definir com seran les usuaris potencials. Un cop definits, serviran per a poder extreure com interaccionaran amb el producte, quines necessitats tindran i en conseqüència els requeriments del producte, tal com afirma Rich Fulcher \cite{user_centred_design}.

Dins de l'anàlisi hi ha quatre subactivitats o passos a seguir:

\subsubsection{Investigació contextual}\label{subsec:investigacio_contextual}
Durant la investigació contextual s'estudia com les persones treballen o interactuen amb el producte en el seu entorn de treball. Per treball s'entén l'ús del producte en si i per entorn de treball, l'entorn en que habitualment s'usa aquest. S'utilitzen aquests termes independentment de la tipologia del producte. És a dir, encara que el producte fos un joc, al fet d'utilitzar-lo se l'anomena treballar. 

Durant la investigació contextual es tracta d'investigar i descobrir com l'usuari treballa en l'entorn habitual i això no es pot determinar enquestant als usuaris. El problema és que la descripció que pugi fer un usuari de com treballa no és fiable. La forma correcta d'investigar és observant com els usuaris treballen i entrevistant-los mentre ells duen a terme aquesta activitat. Per tant es tracta de:

\begin{itemize}
\item Preparar i realitzar visites de camp a l'entorn de treball, on el producte serà utilitzat, de l'usuari/client.
\item Observar i entrevistar els usuaris mentre treballen.
\item Indagar en l'estructura de la pròpia pràctica de treball de l'usuari.
\item Aprendre com la gent treballa en l'entorn en el qual treballarà el producte a dissenyar.
\item Extreure notes detallades de les observacions i entrevistes.
\end{itemize}

Durant la investigació contextual és important no preguntar als usuaris que volen o necessiten. En aquesta etapa no es busca que necessiten sinó observar i entrevistar els usuaris en el seu entorn de treball sobre com treballen.


\subsubsection{Anàlisi contextual}\label{subsec:analisi_contextual}
L'essència d'aquest pas és el processament, la interpretació i l'anàlisi de la informació aconseguida a la investigació contextual (apartat \ref{subsec:investigacio_contextual}). Això s'aconsegueix a base de:

\begin{itemize}
\item Crear un model de flux.
\item Sintetitzar la informació en \glspl{workActivityNotes}.
\item Construir un \ac{WAAD} a partir de les \glspl{workActivityNotes}.
\end{itemize}
%TODO Vocabulari: notes d'activitats de treball??

El model de flux és una representació gràfica que explica com les diferents entitats es comuniquen per tal d'aconseguir que el treball es realitzi. Per a poder crear el model de flux cal identificar els rols de treball. Un rol de treball és una col·lecció de responsabilitats que desenvolupen una part coherent del treball.

\begin{figure}[htp]
\centering
\includegraphics[scale=0.6]{flow_model_example.png}
\caption{Exemple de model de flux.}\label{fig:flow_model_example}
\end{figure}

Paral·lelament a la creació del model de flux, cal sintetitzar la informació en brut que s'ha extret a la investigació contextual. Això es fa creant \glspl{workActivityNotes} les quals, un cop tota la informació ha estat processada, han de representar tota la informació abans extreta. Aquestes notes es caracteritzen per estar escrites en primera persona (des de la perspectiva de l'usuari) parafrasejant i sintetitzant la opinió d'aquest. Cada nota ha de ser concisa i compacta, de manera que expressi una sola idea. Un exemple d'aquestes notes és el de la figura \ref{fig:workActivityNote1}. Com es pot veure cal etiquetar les notes amb un identificador representant l'usuari del qual provenen.

\begin{figure}[htp]
\centering
\includegraphics[scale=0.3]{WorkActivityNotes1.png}
\caption{Exemple d'una nota d'activitats de treball}\label{fig:workActivityNote1}
\end{figure}

Les \glspl{workActivityNotes} serveixen per a construir el \ac{WAAD}. Aquest diagrama consisteix en l'agrupació de les notes segons grups o afinitats segons la perspectiva de l'usuari. L'objectiu d'aquest diagrama és transmetre de forma clara i ràpida la opinió dels usuaris. El que es busca es que ja no sigui necessari llegir les llargues transcripcions de la investigació contextual ja que el \ac{WAAD} n'és una representació d'aquesta. 

%TODO Imatge representant-lo?

\subsubsection{Extracció dels requeriments d'interacció}\label{subsubsec:Extraccio_requeriments}
La idea general d'aquesta etapa es recórrer l'estructura jeràrquica del \ac{WAAD} per extreure sentencies sobre els requeriments del sistema. Això és dur a terme analitzant les \glspl{workActivityNotes} per deduir les necessitats i/o requeriments que cada nota implica. Els requeriments que s'extreuen s'han d'etiquetar per categories (i subcategories si fa falta) juntament amb un identificador que els relacioni amb la \gls{workActivityNotes} de la qual prové. Així si en un anàlisi posterior sorgeixen dubtes, es pot buscar la font de cada requeriment. 

És també important extreure aquells requeriments que l'usuari considera obvis i que per tant no menciona ni descriu i que per tant no estan implícitament a les \glspl{workActivityNotes}.

\begin{figure}[htp]
\centering
\includegraphics[scale=0.3]{WorkActivityNotes2.png}
\caption{Exemple d'extracció de requeriments}\label{fig:workActivityNote2}
\end{figure}

A la figura \ref{fig:workActivityNote2} es pot veure com s'extreu un requeriment, utilitzant el mateix exemple que abans (figura \ref{fig:workActivityNote1}). L'etiqueta "C2", fa referencia a la posició que ocupava la nota dins el \ac{WAAD}. S'utilitzen les lletres per anomenar les diferents branques i sub-branques i els números per diferenciar les notes que hi ha la mateixa branca del \ac{WAAD}.
%TODO Especificar millor que vol dir C2.

Un cop generats els requeriments es comprovarà que aquests siguin correctes preguntant directament als usuaris. Sempre que sigui possible es preguntarà als usuaris que van participar en la investigació contextual (apartat \ref{subsec:investigacio_contextual} juntament amb d'altres nous usuaris. Aquest pas també pot servir perquè els usuaris ajudin a destacar aquells requeriments que són prioritaris.

\subsubsection{Construcció de models informatius per al disseny}\label{subsubsec:Construccio_models}
Per dur a terme aquesta etapa també cal recórrer el \ac{WAAD}, és per això que aquesta etapa no és posterior a l'etapa \ref{subsubsec:Extraccio_requeriments} sinó que les dues es duen a terme de forma paral·lela. L'objectiu d'aquesta etapa és obtenir una sèrie de documents que descriuen tant el sistema actual, com el sistema que es preveu. Aquests documents seran els que es faran servir per a dissenyar el nou producte.

Aquest pas, juntament amb l'anterior (apartat \ref{subsec:analisi_contextual}) serveixen de pont entre l'anàlisi en si i l'etapa del disseny. És a dir, serveixen per enllaçar la situació o model actual, amb el model o sistema que s'està dissenyant.

Els documents que s'obtenen en aquesta etapa (figura \ref{fig:design-informing_models}) són: 
\begin{figure}[ht]
\centering
\includegraphics[scale=0.75]{Design-informing_models.png}
\caption{Exemple d'extracció de requeriments}\label{fig:design-informing_models}
\end{figure}

\begin{description}
\item[Rols de treball] Corresponen als deures, funcions i activitats que desenvolupa una persona amb cert lloc de treball.
\item[Classes d'Usuaris] Són les diferents característiques de la gent que desenvolupa un rol de treball concret.
\item[Model social] És un diagrama que mostra l'organització i relació que existeix entre les diferents persones que intervenen en el sistema.
\item[Model de flux] Aquest diagrama mostra com les diferents entitats (ja siguin, persones, aparells o programes) interaccionen entre si i què intercanvien entre elles.
\item[Inventari de tasques jeràrquic] Es tracta d'un inventari jeràrquic que mostra les diferents tasques que es poden executar en el sistema.
\item[Models d'interacció de les tasques] És un document que detalla com es duen a terme les tasques i com interaccionen les entitats que intervenen (sempre que intervingui més d'una entitat).
\item[Model d'artefacte] Aquest diagrama mostra com els diferents elements tangibles interactuen entre si.
\item[Model físic] Aquest model mostra la distribució física dels diferents artefactes i entitats.
\item[Recopilatori de barreres] És un recopilatori de les barreres que s'han descrit als documents anteriors.
\end{description}

Una barrera és un problema que interfereix amb les operacions que l'usuari executa normalment. És qualsevol cosa que impedeix l'activitat de l'usuari, interromp el flux habitual del treball o interfereix amb el desenvolupament del treball. Seguint les recomanacions de Rex Harton (2012, p. 186) \cite{UX_Book} s'utilitzarà la mateixa simbologia que Beyer i Holtzblatt \cite{Contextual_Design} per a representar les barreres (el llamp vermell \barrier)

\subsection{Disseny}
Aquest pas consisteix en crear diversos dissenys conceptuals que mostren com serà el producte que es busca crear, determinant com ha de ser la interacció amb aquest i la seva aparença. 

Al dissenyar és important saber per a quin tipus d'usuari es dissenya. I és que tal com diu Cooper (2004, p. 124) \cite{Cooper} no és possible crear un disseny que funcioni per a tothom i és millor tenir un petit percentatge de la població completament satisfeta que no pas tota la població mig satisfeta. Afirma que fins i tot és preferible tenir un percentatge més petit de la població extasiada amb el producte. Per a facilitar que en aquesta etapa es dissenyi per a satisfer totes les necessitats de cert grup de la població, cal crear personatges, un personatge per a cada rol definit prèviament. La creació d'aquests personatges és un pas clau per a aconseguir crear un bon producte. Per a crear cada personatge primer es creen personatges a partir dels usuaris que han intervingut a l'anàlisi. Després es fusionen els personatges que tenen les mateixes metes. Ara, d'entre els personatges que queden, cal escollir aquell personatge que, si es dissenya exclusivament per a ell, el producte funcionarà prou bé per la resta de persones. Si cal s'agafaran característiques de diferents personatges per a crear el personatge definitiu. 

Un cop es tenen els personatges creats, comença la part del disseny en si. Aquí Rex (2012, p. 335) \cite{UX_Book} proposa fer 5 passos on cada cop es refina més el disseny, fins a assolir el disseny definitiu. Aquestes etapes són:

\begin{itemize}
\item Ideació i esbossos
\item Disseny conceptual
\item Disseny intermedi
\item Disseny detallat
\item Refinat del disseny
\end{itemize}


\subsubsection{Ideació i esbossos}
Aquest primer pas consisteix en explorar idees. Per una banda amb la ideació, és a dir, el procés de formar idees per el disseny, la qual cosa normalment és fa amb una pluja d'idees. Per l'altre es creen esbossos per a plasmar les idees d'alguna de les persones que està dissenyant. Es important remarcar que les idees o esbossos que es creen en aquest pas han de tenir un nivell de detall molt baix. El que es busca és un primer contacte amb les idees, per tant és important deixa llibertat per a que sorgeixin el màxim d'idees, sense limitar-les a si semblen factibles o no. 

\subsubsection{Disseny conceptual}
Per \gls{conceptual_design} s'entén, un tema, noció o idea amb el propòsit de comunicar una visió del disseny del sistema o producte. És la part del disseny del sistema que porta el model mental del dissenyador a la vida. 
Aquí es busca avaluar i comparar diversos dissenys conceptuals mirant també la seva viabilitat. 

\subsubsection{Disseny intermedi}
L'objectiu del disseny intermedi és crear la navegació, estructura i disseny de les pantalles, amb un nivell de fidelitat mitjà. Parteix dels dissenys conceptuals i es busca descomposar-los en unitats lògiques, expandint-les en diferents possibilitats de disseny. 

En el disseny intermedi és defineix completament com ha de ser la navegació. Tot i això el contingut de cada apartat o pàgina només es mostra de forma aproximada, amb un nivell de fidelitat mitjà, sense masses detalls. 
%TODO parlar dels Design-informing models (p. 375)

\subsubsection{Disseny detallat}
En aquest pas es busca detallar el disseny. Aquí es defineix completament l'aparença de tots els elements que apareixen en pantalla, definint els objectes que els formen, colors, mides, fonts, marges i localització de cada un d'ells. 

%TODO imatge sunshine?

\subsubsection{Refinat del disseny}
Aquesta etapa es centra a buscar i arreglar problemes de \ac{UX} 


%TODO HEX proposa que s'adaptin els metodes als recursos disponibles
%TODO Pagina 161 - 196 5.0.0
\subsection{Implementació}
En la implementació o prototipatge es busca poder avaluar amb els usuaris els dissenys que s'han creat i es du a terme mitjançant prototips. Tal com Nielsen (1993 \ref{Nielsen_1993}) proposa els prototips es poden classificar segons la funcionalitat i segons les funcions o característiques que implementen, tal com es veu a la figura \ref{fig:types_prototypes}. També defineix els prototips horitzontals, verticals i locals. Aquests es caracteritzen per:

\begin{figure}[ht]
\centering
\includegraphics[scale=0.8]{Types_prototypes.png}
\caption{Tipus de prototips}\label{fig:types_prototypes}
\end{figure}

\begin{description}
\item[Prototip Horitzontal] És caracteritza per tenir moltes funcions però amb una profunditat molt baixa de funcionalitat. Serveix per a demostrar les diferents característiques o funcions que tindrà el producte, de manera que es pot veure de forma general les seccions o apartats i la navegació en general.
\item[Prototip Vertical] Aquest tipus de prototip té una profunditat màxima de funcionalitat però centrat només en unes poques funcions. S'utilitza per avaluar amb suficient detall algunes funcions concretes del producte. 
\item[Prototip Local] És un tipus de prototip on només implementa unes poques funcions i amb poca profunditat en quan a funcionalitat. S'utilitzen per avaluar diferents alternatives de disseny per a certs detalls d'interacció amb el sistema. 
\end{description}

Els prototips també es poden classificar segons el nivell de fidelitat, és a dir, com de "finalitzat" percep l'usuari el prototip. Es classifiquen doncs en baixa, mitja o alta fidelitat.  

\begin{description}
\item[Prototips de baixa fidelitat] Un prototip de baixa fidelitat serà percebut com a poc fidel al producte final per un usuari, i com a conseqüència, servira per desinhibir als usuaris a l'hora de criticar-lo i aportar idees per a millorar-lo. És per això que s'utilitzen prototips de baixa fidelitat en les primeres etapes de disseny, així els usuaris tenen absoluta llibertat per a aportar idees.
\item[Prototips de mitja fidelitat]
\item[Prototips d'alta fidelitat] Un prototip d'alta fidelitat serà molt realista i pròxim al producte final. Al ser tant pròxim al producte final, l'usuari percep que hi ha molta feina darrere d'aquest tipus de prototip i que per tant és complicat modificar-lo i això portarà a que tingui a no criticar-lo, amb la idea de no menystenir la feina dels demés. 
\end{description}


\subsection{Avaluació}

\subsection{Pas a pas}
\begin{tabular}{ | p{1.8cm} | p{5cm} | p{3cm} | p{3cm} |}
\hline
\textbf{Etapa} & \textbf{Propòsit} & \textbf{Prototip} & \textbf{Avaluació} \\
\hline
Ideació i esbossos & Explorar idees de disseny & Esbossos & Discutint i criticant \\
\hline
Disseny conceptual & Avaluar i comprar múltiples dissenys conceptuals & Prototips de paper i \textit{wireframes} i \textit{storyboards} de baixa fidelitat &  Storytelling to Key stakeholders\\ %TODO 
\hline
Disseny intermedi & Filtrar els dissenys conceptuals, tot definint la navegació, fins a arribar al disseny conceptual definitiu & \textit{wireframes} d'alta fidelitat & Validar amb els usuaris\\
\hline
Disseny detallat & Definir completament el disseny, definint amb detall l'aparença, la distribució i comportament de les pantalles & \textit{wireframes} d'alta fidelitat i maquetes o prototips interactius & ? \\
\hline
Refinat del disseny & Avaluar el disseny final alhora que trobar i eliminar el màxim de problemes de \ac{UX} & Prototip programat d'alta fidelitat& Rapid method or full rigorous (a definir)\\ %TODO
\hline
\end{tabular}