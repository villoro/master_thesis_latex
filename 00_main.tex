\documentclass{book}

\usepackage[utf8]{inputenc} %caràcters en català
\usepackage[printonlyused]{acronym} %acrònimcs
\usepackage{hyperref} %hiperlinks
\usepackage{glossaries} %glossaris
\usepackage{enumitem} %per enumerar i fer llistes
\usepackage{graphicx} %incloure figures
\usepackage{geometry} %marges de la pagina
\usepackage{pdfpages} %incloure pdfs dins el document
\usepackage{eurosym} \DeclareUnicodeCharacter{20AC}{\euro} %pel simbol €
\usepackage{setspace} %Per l'interlineat
\usepackage{subcaption} %Per posar imatges una al costat de l'altre
\usepackage{fancyhdr} %Capçalera i Peu de pagina maco

\newcommand*\barrier{\includegraphics[scale=0.5]{red_lightning.png}} %simbol llamp
\newcommand*\blacktriangle{\includegraphics[scale=0.05]{blacktriangle.png}}

%Mides document
\geometry{
 a4paper,
 total={210mm,297mm},
 left=30mm, %interior
 right=20mm, %exterior 
 top=25mm,
 bottom=25mm,
 includehead, includefoot %Incloure capçalera i peu dins dels marges 
}

%Format de les pàgines d'inici de capítol
\fancypagestyle{plain}{
	\fancyhf{} % clear all header and footer fields
	\fancyhead[LE, RO]{\thepage} %Part exterior
	\fancyhead[LO, RE]{Estudi de l'UX sobre les aplicacions de gestió de despeses personal per Android} %Part interior
	\renewcommand{\footrulewidth}{0pt}
	\fancyfoot[LE, RO]{\includegraphics[height=1.5cm]{logo_etseib.png}} %Logo
	\textheight = 225mm %Deixar espai pel logo reduint espai pel text
	\footskip = 22mm %Deixar espai pel logo augmentant el peu de pagina
}

%Format de la capçalera i del peu pàgina
\pagestyle{fancy}
	\fancyhead[LE, RO]{\thepage} %Part exterior
	\fancyhead[LO, RE]{Estudi de l'UX sobre les aplicacions de gestió de despeses personal per Android} %Part interior
%\fancyfoot[LE, RO]{\leftmark} %Part exterior
%\fancyfoot[LO, RE]{Arnau Villoro Bort} %Pert interior
	\fancyfoot[C]{} %Eliminar part central
	\fancyfoot[LE, RO]{\includegraphics[height=1.5cm]{logo_etseib.png}} %Logo
	\textheight = 225mm %Deixar espai pel logo reduint espai pel text
	\footskip = 22mm %Deixar espai pel logo augmentant el peu de pagina


%Noms de les parts
\renewcommand{\chaptername}{Capítol}
\renewcommand{\contentsname}{Sumari}
\renewcommand{\listtablename}{Llista de taules}
\renewcommand{\figurename}{Figura}
\renewcommand{\tablename}{Taula} 

%Colors a les taules
\usepackage{xcolor, colortbl}
\definecolor{blue_table_1}{RGB}{220,230,241}
\definecolor{blue_table_2}{RGB}{141,180,226}
\definecolor{blue_table_3}{RGB}{83,141,213}

%Mides i format dels paràgrafs i del text
\setlength{\parindent}{1em} %Paragraph Indentation: espai horitzontal previ a l'inici d'un paragraf
\setlength{\parskip}{1em} %Paragraph spacing: espai entre paragraphs
\onehalfspace %Line spacing: interlineat

%Exemples macos: http://tex.stackexchange.com/questions/1319/showcase-of-beautiful-typography-done-in-tex-friends

\title{Estudi de l'experiència d'usuari sobre les aplicacions de gestió de despeses personals per Android}
\author{Arnau Villoro Bort}

\makeglossaries

%------------------------------------------------------------------
\begin{document}

\newcommand{\blueA}{\cellcolor{blue_table_1}}
\newcommand{\blueB}{\cellcolor{blue_table_2}}
\newcommand{\blueC}{\cellcolor{blue_table_3}}
\newcommand{\headA}[1]{\multicolumn{1}{|c|}{\blueA \textbf{#1}}}
\newcommand{\headB}[1]{\multicolumn{1}{|c|}{\blueB \textbf{#1}}}
\newcommand{\headC}[1]{\multicolumn{1}{|c|}{\blueC \textbf{#1}}}
\newcommand{\noBorde}[1]{\multicolumn{1}{#1}{}}


\frontmatter

\includepdf{Portada.pdf}

%\maketitle

\chapter*{Resum} \label{sec:Resum}
\addcontentsline{toc}{chapter}{Resum}

En aquest projecte s'ha fet un estudi d'Experiència d'Usuari (UX) sobre les aplicacions de gestió de despeses personals per a \gls{Android}. Per a tal fi, en primer lloc s'ha definit què és i com es fan els estudis d'UX, explicant amb detall en que consisteixen les seves 4 etapes, que són: Anàlisi, Disseny, Implementació i Avaluació.

L'estudi en si ha començat analitzant com els usuaris gestionen les seves despeses i la seva satisfacció amb les aplicacions existents més rellevants per a \gls{Android}. Utilitzant aquest anàlisi s'ha definit els requeriments necessaris per a una aplicació per a la gestió de les despeses personals. Després d'aquest anàlisi previ s'ha fet un procés iteratiu on es dissenyava i s'implementaven prototips d'aquesta aplicació, augmentant-ne la fidelitat i el detall d'aquests a cada iteració. A continuació s'utilitzaven aquests prototips amb usuaris per a validar que satisfeien les seves necessitats, modificant els requeriments inicials si era necessari. 

Com a resultat d'aquest procés iteratiu s'ha creat un prototip programat que implementa les funcions més importants d'una aplicació per a gestionar les despeses domèstiques. Llavors s'ha comparat l'experiència d'usuari que proporcionava aquest prototip respecte les aplicacions ja existents al mercat. El resultat és un prototip que, proporciona la millor UX d'entre les aplicacions ja existents.

Per a poder dissenyar aquest prototip a calgut modelitzar el problema de la resolució dels deutes en grup, alhora que dissenyar i implementar un algoritme que ho solucioni. També s'ha comprovat el seu cost d'execució amb diversos jocs de dades. 

Finalment s'ha descrit els passos que caldria seguir per a poder passar del prototip programat a una aplicació completament funcional que es pugui penjar a les botigues d'aplicacions. 




\tableofcontents

\chapter*{Llista d'acrònims}
\label{sec:glossary}
\addcontentsline{toc}{chapter}{Glossari}

%INFO: http://en.wikibooks.org/wiki/LaTeX/Glossary

\begin{acronym}
\acro{PFC}{Projecte Final de Carrera}
\acro{UX}{User Experience}
\acro{UI}{User Interface}
\acro{CSV}{Comma-Separated Values}
\acro{WAAD}{Work Activity Affinity Diagram}
\end{acronym}


\newglossaryentry{Android}
{
name = Android, description = Sistema operatiu mòbil amb el qual funcionen la majoria de telèfons mòbils \cite{Android_OS}
}

\newglossaryentry{conceptual_design}
{
name = disseny conceptual, plural = dissenys conceptuals, description = {és un tema, noció o idea amb el propòsit de comunicar una visió del disseny del sistema o producte. És la part del disseny del sistema que porta el model mental del dissenyador a la vida.}
}

\newglossaryentry{Google_play}
{
name = Google Play, description = Botiga virtual de Google en la qual es troben les aplicacions per a dispositius mòbils que funcionen amb \gls{Android} (\url{https://play.google.com/store/apps})
}

\newglossaryentry{Holo}
{
name = Holo, description = són unes directrius de disseny per a les aplicacions d'\gls{Android} que es van crear amb la versió 4.0
}

\newglossaryentry{Material}
{
name = Material Design, description = són unes directrius de disseny per a les aplicacions d'Android que es van crear amb la versió 5.0 per a substituir i millorar Holo. Per més informació \url{http://developer.android.com/design/material/index.html}
}

\newglossaryentry{workActivityNotes}
{
name = nota d'activitats de treball, plural = notes d'activitats de treball, description = es tracta de notes que parafrasegen i representen la opinió d'un usuari per a facilitar la comprensió de la opinió dels usuaris.  
}

%TODO Prototip vertical, horitzontal, T (p91)
%TODO UX

\newglossaryentry{smartphone}
{
name = \textit{smartphone}, description = Telefon intel·ligent que permeten realitzar tasques semblants a les realitzades per ordinadors 
}


%\newglossaryentry{persona}
%{
%name = personatge, description = és
%}

\printglossary


\chapter*{Apps}
\label{sec:apps}

\begin{tabular}{ | l | c | l | l | }
\hline
\textbf{Núm.} &  & \textbf{Nom} & \textbf{Autor} \\
\hline
App 1 & \includegraphics[scale=0.05]{A01_icon.png} & \href{https://play.google.com/store/apps/details?id=com.expensemanager}{Expense Manager} & Bishinews \\

App 2 & \includegraphics[scale=0.05]{A02_icon.png} & \href{https://play.google.com/store/apps/details?id=at.markushi.expensemanager}{Expense Manager} & Markus Hintersteiner \\

App 3 & \includegraphics[scale=0.05]{A03_icon.png} & \href{https://play.google.com/store/apps/details?id=com.bruno.myapps.droidwallet}{Droid Wallet} & William Bruno \\

App 4 & \includegraphics[scale=0.05]{A04_icon.png} & \href{https://play.google.com/store/apps/details?id=com.code44.finance}{Financius - Expense Manager} & Mantas Varnagiris \\

App 5 & \includegraphics[scale=0.05]{A05_icon.png} & \href{https://play.google.com/store/apps/details?id=com.handyapps.expenseiq}{Expense IQ} & Handy Apps \\

App 6 & \includegraphics[scale=0.05]{A06_icon.png} & \href{https://play.google.com/store/apps/details?id=com.techahead.ExpenseManager}{Diario Gasto Gerente (Daily Expense Manager)} & Gullak \\

App 7 & \includegraphics[scale=0.05]{A07_icon.png} & \href{https://play.google.com/store/apps/details?id=com.bookmark.money}{Money lover - Expense Manager} & ZooStudio   \\

App 8 & \includegraphics[scale=0.05]{A08_icon.png} & \href{https://play.google.com/store/apps/details?id=com.kpmoney.android}{AndroMoney (Expense Track)} & AndroMoney \\

App 9 & \includegraphics[scale=0.05]{A10_icon.png} & \href{https://play.google.com/store/apps/details?id=cz.destil.settleup}{Settle up} & David Vávra \\

App 10 & \includegraphics[scale=0.05]{A11_icon.png} & \href{https://play.google.com/store/apps/details?id=com.Splitwise.SplitwiseMobile}{Splitwise} & Splitwise \\

App 11 & \includegraphics[scale=0.2]{A12_icon.png} & Expensor & Arnau Villoro \\
\hline
\end{tabular}
\chapter{Prefaci} \label{Prefaci}

Des que tinc memòria he estat molt interessat en la gestió de la informació i de les dades, així com l'enregistrament d'aquestes. No és estrany doncs, el meu interès per a gestionar i controlar les meves despeses. Amb l'aparició dels \glspl{smartphone}, o telèfons intel·ligents, portar un registre de despeses va passar a ser quelcom bastant fàcil. Només calia descarregar-se una aplicació per al mòbil i amb aquesta podies fàcilment apuntar totes les despeses. 

Més endavant vaig descobrir que també hi havia aplicacions que permetien dividir fàcilment les despeses fetes en grup, per a gestionar, per exemple, un viatge amb els amics. Però amb el temps em vaig adonar que, tot i provar-ne moltes, no hi havia cap aplicació que satisfés les meves necessitats. Va ser per això que, cap al Octubre del 2013, vaig decidir que havia de crear jo la meva aplicació.

El problema va ser que en aquell moment em faltaven molts coneixements, però el que em mancava, ho compensava amb moltes ganes i il·lusió. Mig any després i amb moltes hores invertides, ja havia aprés a usar el llenguatge Java, així com a fer aplicacions per a \gls{Android}. 

Paral·lelament vaig començar a buscar idees sobre que podia fer el meu \ac{PFC}. Fins que un bon dia vaig veure unes propostes del Sergi (el meu tutor del \ac{PFC}) a la borsa de projectes, sobre coses relacionades amb \gls{Android} i aplicacions, i que estava obert a propostes. 

A partir d'aquí només va caldre una reunió per a entendre'ns, i així va ser com vam començar aquest projecte. 






\mainmatter
\chapter{Introducció}

\section{Objectius del projecte}\label{objectius}
L'objectiu d'aquest \ac{PFC} és fer un estudi de l'Experiència d'Usuari o \ac{UX} per una aplicació que serveixi per a gestionar les despeses personals domèstiques. Prèviament a l'estudi, es busca definir com es fan actualment els estudis \ac{UX} i quines parts tenen aquests tipus d'estudis. Després es procedirà amb l'estudi en si, on es busca definir com ha de ser una aplicació d'aquest tipus per a que garanteixi una bona \ac{UX}.

Concretament es busca dissenyar una aplicació que:
\begin{itemize}
\item Permeti enregistrar despeses i ingressos tot categoritzant-los.
\item Serveixi per a recordar els deutes (positius o negatius) que es tenen amb diverses persones.
\item Faciliti la gestió de despeses grupals alhora que permeti saldar els deutes minimitzant les transaccions entre els membres.
\item Sigui intuïtiva i senzilla de fer servir.
\item Visualment sigui agradable i minimalista per a que sigui agradable i còmode per a l'usuari.
\item Agradi als usuaris en general.
\end{itemize}

Per a dissenyar l'aplicació primer s'analitzarà com interaccionen els usuaris amb les aplicacions de gestió de despeses, per extreure'n les necessitats i les funcionalitats necessàries. Finalment es comprovarà la idoneïtat del disseny creat.

A més, també es busca resoldre els problemes d'enginyeria que es puguin derivar de les diverses funcions que haurà d'implementar l'aplicació. 

\section{Abast del projecte}
En aquest projecte s'estudiarà l'\ac{UX} per al tipus d'aplicacions esmentat tot definint quines utilitats i funcionalitats ha de tenir l'aplicació i com ha de ser la \ac{UI}. L'estudi s'enfocarà únicament en una aplicació per a \glspl{smartphone} que funcionen amb \gls{Android}, l'actual sistema operatiu més utilitzat\cite{Android_OS} en \glspl{smartphone} o telèfons intel·ligents. 
Finalment pel que fa al desenvolupament de l'aplicació, no es considera factible crear-la sencera amb tots els requeriments que es puguin deduir amb l'estudi de l'\ac{UX}. És per això que es faran proves de concepte, creant un o varis prototips que s'apropin el màxim possible a com hauria de ser l'aplicació. 

\chapter{Experiència d'Usuari}
\section{Que s'entén per Experiència d'Usuari}
Segons Rex Hartson (2012, p. 19)\cite{UX_Book}, l'\ac{UX} és la totalitat de l'efecte o efectes que sent (o experimenta) internament l'usuari com a resultat de la interacció, i del context d'ús, amb el sistema, dispositiu o producte. És a dir, una bona \ac{UX} es produirà quan l'usuari gaudeixi interaccionant i utilitzant el dispositiu o producte. Interacció i ús s'empren en un sentit molt ampli, ja que inclouen veure, tocar, pensar sobre el producte/dispositiu o fins i tot admirar-lo. 
A més, l'\ac{UX} també engloba la usabilitat i la utilitat. Certament l'usuari sent internament parts de la usabilitat, com l'augment de productivitat. Però hi ha certes manifestacions de usabilitat, com podria ser el temps invertit en la tasca, que representa un component no necessàriament experimentat internament per l'usuari.

\section{Com s'estudia l'Experiència d'Usuari?}
L'\ac{UX} no pot ser dissenyada ja que depèn, no només del producte en si mateix, sinó que també depèn de l'usuari i la situació en la que l'utilitza [Smashing Magazine, 2012, p. 25-28]\cite{Smashing_User_Experience_Design}. I és que no és possible dissenyar ni l'usuari ni la situació. Però el que si es pot fer és dissenyar per a una bona \ac{UX}. Seguint els passos que proposa Rex Harton en el seu llibre d'\ac{UX} \cite{UX_Book}, per a aconseguir-ho hi ha quatre activitats elementals que són anàlisi, disseny, implementació i avaluació, tal i com es pot veure a la figura \ref{fig:UX_lifecycle}. Per tal d'aconseguir proporcionar una bona \ac{UX} aquestes activitats és duen a terme de forma iterativa, ja que no sempre és possible trobar un bon disseny al primer intent.

\begin{figure}[htp]
\centering
\includegraphics[scale=0.6]{UX_wheel.png}
\caption{Activitats a seguir per a dissenyar garantint una bona \ac{UX}}\label{fig:UX_lifecycle}
\end{figure}

Aquestes quatre activitats, a grans trets, consisteixen en:
\begin{description}
\item [Anàlisi] Es basa en entendre les necessitats de l'usuari que utilitzarà el producte.
\item [Disseny] Consisteix en la creació de dissenys conceptuals determinant la interacció, el comportament i l'aparença del producte.
\item [Implementació] Correspon a la creació del prototip.
\item [Avaluació] Es tracta de comprovar si el disseny satisfà les necessitats dels usuaris que s'han determinat.
\end{description}

\subsection{Anàlisi}\label{sec:analisi}
L'objectiu general d'aquesta activitat és definir com seran les usuaris potencials. Un cop definits, serviran per a poder extreure com interaccionaran amb el producte, quines necessitats tindran i en conseqüència els requeriments del producte, tal com afirma Rich Fulcher \cite{user_centred_design}.

Dins de l'anàlisi hi ha quatre subactivitats o passos a seguir:

\subsubsection{Investigació contextual}\label{subsec:investigacio_contextual}
Durant la investigació contextual s'estudia com les persones treballen o interactuen amb el producte en el seu entorn de treball. Per treball s'entén l'ús del producte en si i per entorn de treball, l'entorn en que habitualment s'usa aquest. S'utilitzen aquests termes independentment de la tipologia del producte. És a dir, encara que el producte fos un joc, al fet d'utilitzar-lo se l'anomena treballar. 

Durant la investigació contextual es tracta d'investigar i descobrir com l'usuari treballa en l'entorn habitual i això no es pot determinar enquestant als usuaris. El problema és que la descripció que pugi fer un usuari de com treballa no és fiable. La forma correcta d'investigar és observant com els usuaris treballen i entrevistant-los mentre ells duen a terme aquesta activitat. Per tant es tracta de:

\begin{itemize}
\item Preparar i realitzar visites de camp a l'entorn de treball, on el producte serà utilitzat, de l'usuari/client.
\item Observar i entrevistar els usuaris mentre treballen.
\item Indagar en l'estructura de la pròpia pràctica de treball de l'usuari.
\item Aprendre com la gent treballa en l'entorn en el qual treballarà el producte a dissenyar.
\item Extreure notes detallades de les observacions i entrevistes.
\end{itemize}

Durant la investigació contextual és important no preguntar als usuaris que volen o necessiten. En aquesta etapa no es busca que necessiten sinó observar i entrevistar els usuaris en el seu entorn de treball sobre com treballen.


\subsubsection{Anàlisi contextual}\label{subsec:analisi_contextual}
L'essència d'aquest pas és el processament, la interpretació i l'anàlisi de la informació aconseguida a la investigació contextual (apartat \ref{subsec:investigacio_contextual}). Això s'aconsegueix a base de:

\begin{itemize}
\item Crear un model de flux.
\item Sintetitzar la informació en \glspl{workActivityNotes}.
\item Construir un \ac{WAAD} a partir de les \glspl{workActivityNotes}.
\end{itemize}
%TODO Vocabulari: notes d'activitats de treball??

El model de flux és una representació gràfica que explica com les diferents entitats es comuniquen per tal d'aconseguir que el treball es realitzi. Per a poder crear el model de flux cal identificar els rols de treball. Un rol de treball és una col·lecció de responsabilitats que desenvolupen una part coherent del treball.

\begin{figure}[htp]
\centering
\includegraphics[scale=0.6]{flow_model_example.png}
\caption{Exemple de model de flux.}\label{fig:flow_model_example}
\end{figure}

Paral·lelament a la creació del model de flux, cal sintetitzar la informació en brut que s'ha extret a la investigació contextual. Això es fa creant \glspl{workActivityNotes} les quals, un cop tota la informació ha estat processada, han de representar tota la informació abans extreta. Aquestes notes es caracteritzen per estar escrites en primera persona (des de la perspectiva de l'usuari) parafrasejant i sintetitzant la opinió d'aquest. Cada nota ha de ser concisa i compacta, de manera que expressi una sola idea. Un exemple d'aquestes notes és el de la figura \ref{fig:workActivityNote1}. Com es pot veure cal etiquetar les notes amb un identificador representant l'usuari del qual provenen.

\begin{figure}[htp]
\centering
\includegraphics[scale=0.3]{WorkActivityNotes1.png}
\caption{Exemple d'una nota d'activitats de treball}\label{fig:workActivityNote1}
\end{figure}

Les \glspl{workActivityNotes} serveixen per a construir el \ac{WAAD}. Aquest diagrama consisteix en l'agrupació de les notes segons grups o afinitats segons la perspectiva de l'usuari. L'objectiu d'aquest diagrama és transmetre de forma clara i ràpida la opinió dels usuaris. El que es busca es que ja no sigui necessari llegir les llargues transcripcions de la investigació contextual ja que el \ac{WAAD} n'és una representació d'aquesta. 

%TODO Imatge representant-lo?

\subsubsection{Extracció dels requeriments d'interacció}\label{subsubsec:Extraccio_requeriments}
La idea general d'aquesta etapa es recórrer l'estructura jeràrquica del \ac{WAAD} per extreure sentencies sobre els requeriments del sistema. Això és dur a terme analitzant les \glspl{workActivityNotes} per deduir les necessitats i/o requeriments que cada nota implica. Els requeriments que s'extreuen s'han d'etiquetar per categories (i subcategories si fa falta) juntament amb un identificador que els relacioni amb la \gls{workActivityNotes} de la qual prové. Així si en un anàlisi posterior sorgeixen dubtes, es pot buscar la font de cada requeriment. 

És també important extreure aquells requeriments que l'usuari considera obvis i que per tant no menciona ni descriu i que per tant no estan implícitament a les \glspl{workActivityNotes}.

\begin{figure}[htp]
\centering
\includegraphics[scale=0.3]{WorkActivityNotes2.png}
\caption{Exemple d'extracció de requeriments}\label{fig:workActivityNote2}
\end{figure}

A la figura \ref{fig:workActivityNote2} es pot veure com s'extreu un requeriment, utilitzant el mateix exemple que abans (figura \ref{fig:workActivityNote1}). L'etiqueta "C2", fa referencia a la posició que ocupava la nota dins el \ac{WAAD}. S'utilitzen les lletres per anomenar les diferents branques i sub-branques i els números per diferenciar les notes que hi ha la mateixa branca del \ac{WAAD}.
%TODO Especificar millor que vol dir C2.

Un cop generats els requeriments es comprovarà que aquests siguin correctes preguntant directament als usuaris. Sempre que sigui possible es preguntarà als usuaris que van participar en la investigació contextual (apartat \ref{subsec:investigacio_contextual} juntament amb d'altres nous usuaris. Aquest pas també pot servir perquè els usuaris ajudin a destacar aquells requeriments que són prioritaris.

\subsubsection{Construcció de models informatius per al disseny}\label{subsubsec:Construccio_models}
Per dur a terme aquesta etapa també cal recórrer el \ac{WAAD}, és per això que aquesta etapa no és posterior a l'etapa \ref{subsubsec:Extraccio_requeriments} sinó que les dues es duen a terme de forma paral·lela. L'objectiu d'aquesta etapa és obtenir una sèrie de documents que descriuen tant el sistema actual, com el sistema que es preveu. Aquests documents seran els que es faran servir per a dissenyar el nou producte.

Aquest pas, juntament amb l'anterior (apartat \ref{subsec:analisi_contextual}) serveixen de pont entre l'anàlisi en si i l'etapa del disseny. És a dir, serveixen per enllaçar la situació o model actual, amb el model o sistema que s'està dissenyant.

Els documents que s'obtenen en aquesta etapa (figura \ref{fig:design-informing_models}) són: 
\begin{figure}[ht]
\centering
\includegraphics[scale=0.75]{Design-informing_models.png}
\caption{Exemple d'extracció de requeriments}\label{fig:design-informing_models}
\end{figure}

\begin{description}
\item[Rols de treball] Corresponen als deures, funcions i activitats que desenvolupa una persona amb cert lloc de treball.
\item[Classes d'Usuaris] Són les diferents característiques de la gent que desenvolupa un rol de treball concret.
\item[Model social] És un diagrama que mostra l'organització i relació que existeix entre les diferents persones que intervenen en el sistema.
\item[Model de flux] Aquest diagrama mostra com les diferents entitats (ja siguin, persones, aparells o programes) interaccionen entre si i què intercanvien entre elles.
\item[Inventari de tasques jeràrquic] Es tracta d'un inventari jeràrquic que mostra les diferents tasques que es poden executar en el sistema.
\item[Models d'interacció de les tasques] És un document que detalla com es duen a terme les tasques i com interaccionen les entitats que intervenen (sempre que intervingui més d'una entitat).
\item[Model d'artefacte] Aquest diagrama mostra com els diferents elements tangibles interactuen entre si.
\item[Model físic] Aquest model mostra la distribució física dels diferents artefactes i entitats.
\item[Recopilatori de barreres] És un recopilatori de les barreres que s'han descrit als documents anteriors.
\end{description}

Una barrera és un problema que interfereix amb les operacions que l'usuari executa normalment. És qualsevol cosa que impedeix l'activitat de l'usuari, interromp el flux habitual del treball o interfereix amb el desenvolupament del treball. Seguint les recomanacions de Rex Harton (2012, p. 186) \cite{UX_Book} s'utilitzarà la mateixa simbologia que Beyer i Holtzblatt \cite{Contextual_Design} per a representar les barreres (el llamp vermell \barrier)

\subsection{Disseny}

%TODO Pagina 161 - 196 5.0.0
\subsection{Implementació}
\subsection{Avaluació}
\chapter{Estat de l'art}
\section{Aplicacions existents}
En quan a les aplicacions que es poden trobar actualment a \gls{Google_play}, la botiga virtual d'aplicacions per \gls{Android}, existeixen moltes que serveixen per a la gestió de despeses. És per això que s'estudiaran només les més rellevants i representatives, les quals tenen un mínim de 100.000 descarregues. 

Actualment hi ha dos tipus d'aplicacions relacionades amb la gestió de despeses. Per una banda les que serveixen per enregistrar les despeses i/o ingressos personals, tot categoritzant-los i per l'altra les que serveixen per a gestionar despeses compartides en grup i/o deutes personals amb coneguts.

\subsection{Aplicacions per enregistrar despeses/ingressos}
La majoria d'aplicacions són d'aquest tipus. Les funcions i característiques que tenen idealment aquest tipus d'aplicacions són:

\begin{itemize}
\item Funcionalitats bàsiques:
\begin{itemize}
\item Enregistrar despeses i ingressos.
\item Crear categories per classificar les despeses/ingressos.
\item Crear despeses/ingressos recurrents al llarg del temps.
\end{itemize}

\item Funcionalitats extres:
\begin{itemize}
\item Crear varis comptes, com podria ser per exemple efectiu i banc.
\item Permeten transaccions entre els diferents comptes
\item Permeten especificar la forma de pagament, com podria ser en efectiu o targeta de crèdit.
\item Utilitzen etiquetes. Aquestes serveixen per a marcar les despeses/ingressos per fer agrupacions diferents a les categories. Per exemple és útil per a marcar totes les despeses fetes durant un viatge independentment de la categoria que siguin. 
\item Incorporen una calculadora per a facilitar la introducció de valors.
\item Permeten crear pressupostos per a les diverses categories.
\item Es pot introduir despeses/ingressos en múltiples divises, com podria ser euros i dòlars.
\item Estan en més d'un idioma.
\end{itemize}

\item Bases de dades:
\begin{itemize}
\item Es pot exportar/importar la base de dades.
\item Permeten exportar/importar en format CSV, cosa que facilita l'edició de les dades externament.
\item Són multidispositiu, es pot instal·lar l'aplicació en varis aparells i les dades són sincronitzades automàticament.
\item Permeten filtrar les dades segons diversos criteris, els quals són personalitzables. 
\end{itemize}

\item Gràfics i informes:
\begin{itemize}
\item Inclouen gràfics amb la distribució percentual de les despeses/ingressos segons les categories.
\item Tenen gràfics que mostren l'evolució temporal.
\item Creen gràfics que mostren l'estat de les despeses d'una categoria concreta respecte el pressupost fixat.
\end{itemize}

\item \ac{UI}
\begin{itemize}
\item La navegació dins l'aplicació és intuïtiva. 
\item La interfase és visualment agradable
\item Segueixen les normes de disseny \gls{Holo} o \gls{Material}. 
\item Inclouen un resum general útil
\item L'aplicació és senzilla d'utilitzar i no es complicat aprendre a fer-la servir.
\end{itemize}
\end{itemize}

El fet és que en realitat de les 7 aplicacions estudiades, cada una d'elles compleix bé en algun apartat, però té manques en els altres tal i com es pot veure al gràfic \ref{fig:taula_resum_apps}.

\begin{figure}[htp]
\centering
\includegraphics[scale=0.5]{grafic_apps.png}
\caption{Caracteristiques i funcions de les aplicacions per enregistrar despeses}\label{fig:taula_resum_apps}
\end{figure}

\subsection{Aplicacions per gestionar despeses en grup}



\include{20_estudi}


\chapter{Parts rellevants de l'aplicació}

\section{Problema del repartiment de despeses}
Una de les funcions que ha de desenvolupar l'aplicació és fer una proposta sobre com solucionar els deutes existents entre els membres d'un grup. Tot i que en un principi es podria intentar resoldre pensant en les transaccions que ha fet cada membre, a l'hora de solucionar els deutes només cal tenir en compte el balanç de cada persona, tal com exposa David Vávra (creador de l'aplicació Settle Up) a la seva Tesis final de Màster (2012, p. 6-7) \cite{Settle_up}.

Un cop s'analitza el problema es pot veure que s'assembla al típic problema del transport, on s'ha de transportar de les fàbriques als magatzems, amb la diferència qualsevol emparellament té un cost nul i el que es busca és minimitzar el nombre de transaccions. 

Agafem com a exemple les següents persones amb els balanços de la taula \ref{table:balances}.

\begin{table}
\centering
\caption{Exemple dels balanços en un grup}
\label{table:balances}
\begin{tabular}{ | l | r |}
\hline
\headB{Persona} & \headB{Balanç} \\
\hline
Arnau & 20,33 \\
\hline
Berta & -3,90 \\
\hline
Càrol & -10,01 \\
\hline
David & 8,57 \\
\hline
Elena & -15,07 \\
\hline
\end{tabular}
\end{table}


\begin{table}
\centering
\caption{Exemple d'una resolució}
\label{table:resolution}
\begin{tabular}{ | r | l | r | r |}
 \cline{3-4}
\noBorde{c} & \noBorde{c|} & 20,33 & 8,57 \\
 \cline{3-4}
\noBorde{c} & \noBorde{c|} & \headB{Arnau} & \headB{David} \\
 \hline
 3,9 & \headB{Berta} & & 3,9\\
 \hline
 10,01 & \headB{Càrol} & 5,32 & 4,69\\
 \hline
 15,01 & \headB{Elena} & 15,01 & \\
 \hline
\end{tabular}
\end{table}


Es pot veure que la Berta, la Càrol i l'Elena vindrien a ser les fàbriques i que l'Arnau i el David els magatzems als quals s'han d'enviar els diners. A la taula \ref{table:resolution} es mostra el problema juntament amb una solució possible.

Cal notar que degut a arrodoniments al calcular quan ha gastat cada persona en cada transacció es possible que la suma de tots els balanços no sigui 0. S'ha optat per beneficiar als creditors, de manera que tots els deutors paguin tot el que els correspon i que els creditors cobrin tot el que han deixat com a mínim, i si s'escau, alguns decimals de més.

David Vávra proposa solucionar el problema minimitzant només el nombre de transaccions total que s'hauran d'efectuar. Per a fer-ho proposa fer servir un mètode heurístic i posteriorment, si el nombre de persones del grup no es molt elevat, buscar la solució òptima. Per a trobar-la calcula totes les possibilitats. Aquesta manera de resoldre no és el més eficient possible. A més, si s'analitza en més profunditat el problema es pot veure que un bon repartiment no només ha de minimitzar el nombre total de transaccions. Entre altres aspectes que es poden tenir en compte, un bon repartiment serà aquell que:

\begin{itemize}
\item Minimitzi el nombre màxim de transaccions totals
\item Minimitzi el nombre màxim de transaccions que ha de fer una sola persona
\item Si es possible emparelli persones que es coneixen entre elles
\end{itemize}

Una manera de resoldre el problema tenint en compte aquests nous objectius és fent ús de la programació lineal. Concretament es té un de problema de \ac{PLEM}.
%TODO glossary

\subsection{Primera modelització del problema amb resolució exacte}
A continuació es modelarà el problema de solucionar els deutes d'un grup. Per a fer-ho només es buscara que minimitzi el nombre màxim de transaccions totals i el de transaccions que ha de fer una sola persona. Si bé, com ja s'ha dit, una bona solució també tindria en compte l'afinitat entre persones, això és més difícil de modelar i de descobrir sense forçar als usuaris a introduir molta informació. Tot i això la modelització serà prou flexible per a que si en un futur es vol afegir aquest objectiu, és pugui. 
\subsubsection{Dades}
\begin{description}
\item NC = nombre de creditors
\item ND = nombre de deutors
\item $D_{i}=$ diners que deu la persona \textbf{i} $\in (1 \ldots ND)$  [€].
\item $C_{j}=$ diners que li deuen a la persona \textbf{j} $\in (1 \ldots NC)$  [€].
\end{description}


\subsubsection{Variables}
\begin{description}
\item $x_{ij}=$ diners que dona la persona \textbf{i} $\in (1 \ldots ND)$ a la persona \textbf{j} $\in (1 \ldots NC)$  [€].
\item $p_{ij}=$ binaria que indica si la persona \textbf{i} $\in (1 \ldots ND)$ paga a la persona \textbf{j} $\in (1 \ldots NC)$.
\item a = màxim de transaccions dels deutors.
\item b = màxim de transaccions dels creditors
\end{description}

\subsubsection{Restriccions}
\begin{tabular}{l l}
$\sum\limits_{\forall j} x_{ij} \geq D_{i} \quad \forall i$ & Cada deutor ha de pagar com a mínim el que li correspon \\

$\sum\limits_{\forall i} x_{ij} \geq C_{j} \quad \forall j$ & Cada creditor ha de rebre com a mínim el que li correspon \\

$x_{ij} \leq M \cdot p_{ij} \quad \forall i \forall j$ & Forçar valor de $p_{ij}$ (M valor suficientment gran, per exemple $M=\sum\limits_{\forall i} D_{i} + \sum\limits_{\forall j} C_{j})$\\

$\sum\limits_{\forall j} p_{ij} \leq a \quad \forall i$ & Forçar valor de $a$ \\

$\sum\limits_{\forall i} p_{ij} \leq b \quad \forall j$ & Forçar valor de $b$ \\
\end{tabular}

\subsubsection{Funció Objectiu}
[min] $z = \sum\limits_{\forall i \forall j} p_{ij} + \lambda \cdot (a+b)$
Es busca minimitzar el nombre total de transaccions, així com el nombre de transaccions màximes que farà una sola persona
\subsubsection{Paràmetres}
$\lambda$ = usat per decidir el pes a la funció objectiu de cada part

\subsection{Modelització millorada amb resolució exacte}
La modelització anterior en alguns casos no trobava una solució factible, degut als arrodoniments a l'hora de calcular els balanços (pas previ a la solució del \ac{PLEM}).
La idea general d'aquesta modelització es calcular els diners que un deutor pagà a un creditor que formen part del deute per una banda i per l'altre els diners extra que ha de pagar (tot i que no li corresponen). Aquests diners extra sortiran a la funció objectiu amb una penalització molt elevada per tal que al resoldre el \ac{PLEM} només siguin més grans de zero en cas de ser necessari. Vist per la banda dels creditors es farà el mateix, es calcularà els diners extra que han de rebre per tal que els deutors paguin el necessari. De fet aquestes dues variables corresponen a les variables de marge associades a la primera i segona restricció respectivament.

\subsubsection{Dades}
\begin{description}
\item $x_{ij}=$ diners que dona la persona \textbf{i} $\in (1 \ldots ND)$ a la persona \textbf{j} $\in (1 \ldots NC)$  [€].
\item $p_{ij}=$ binaria que indica si la persona \textbf{i} $\in (1 \ldots ND)$ paga a la persona \textbf{j} $\in (1 \ldots NC)$.
\item a = màxim de transaccions dels deutors.
\item b = màxim de transaccions dels creditors
\end{description}

\subsubsection{Variables}
\begin{description}
\item $x_{ij}=$ diners que dona la persona \textbf{i} $\in (1 \ldots ND)$ a la persona \textbf{j} $\in (1 \ldots NC)$  [€].
\item $p_{ij}=$ binaria que indica si la persona \textbf{i} $\in (1 \ldots ND)$ paga a la persona \textbf{j} $\in (1 \ldots NC)$.
\item a = màxim de transaccions dels deutors.
\item b = màxim de transaccions dels creditors 
\item $t_{i}=$ diners extres que ha de pagar la persona \textbf{i} $\in (1 \ldots ND)$  [€].
\item $q_{j}=$ diners extres que ha de rebre la persona \textbf{j} $\in (1 \ldots NC)$  [€].
\end{description}

\subsubsection{Restriccions}
\begin{tabular}{l l}
$\sum\limits_{\forall j} x_{ij} + t_{i} = D_{i} \quad \forall i$ & Cada deutor ha de pagar com a mínim el que li correspon \\

$\sum\limits_{\forall i} x_{ij} + q_{j} = C_{j} \quad \forall j$ & Cada creditor ha de rebre com a mínim el que li correspon \\

$x_{ij} \leq M \cdot p_{ij} \quad \forall i \forall j$ & Forçar valor de $p_{ij}$ (M valor suficientment gran, per exemple $M=\sum\limits_{\forall i} D_{i} + \sum\limits_{\forall j} C_{j})$\\

$\sum\limits_{\forall j} p_{ij} \leq a \quad \forall i$ & Forçar valor de $a$ \\

$\sum\limits_{\forall i} p_{ij} \leq b \quad \forall j$ & Forçar valor de $b$ \\
\end{tabular}

\subsubsection{Funció Objectiu}
[min] $z = \sum\limits_{\forall i \forall j} p_{ij} + \lambda_{1} \cdot \sum\limits_{\forall i} t_{i} + \lambda_{2} \cdot \sum\limits_{\forall j} q_{j} + \lambda_{3} \cdot a + \lambda_{4} \cdot b $

Es busca minimitzar el nombre total de transaccions, així com el nombre de transaccions màximes que farà una sola persona
\subsubsection{Paràmetres}
$\lambda$ = usat per decidir el pes a la funció objectiu de cada part

\subsubsection{Post-procés}
Finalment, per calcular que ha de pagar cada persona es repartiran els diners extres que han de rebre els creditors per tal que tots ells rebin com a mínim tot el que es deu. No es repartiran els diners extres que han de pagar els deutors ja que amb què els creditors rebin el que se'ls hi devia ja n'hi ha prou. Aquests diners extres que un creditor ha de rebre es repartiran entre totes les persones que han de donar-li diners, deixant de banda els deutors que no han de pagar res a aquell creditor. 

Per tant el deutor \textbf{i} pagarà al creditor \textbf{j}: $\frac{x_{ij} + q_{j}*p_{ij}}{\sum\limits_{\forall k} p_{ik}}$

\subsection{Estudi del temps computacional de la modelització millorada}
Un cop es té una bona modelització del problema, cal provar-la amb un aparell i dades reals per comprovar que el temps necessari per resoldre el problema no sigui excessiu. Per a fer-ho s'ha utilitzat un \gls{Nexus_5} i la llibreria per a resoldre problemes de programació lineal \gls{lp_solve}.

En un primer moment s'ha fet servir les dades de 17 balanços provinent de grups de despeses reals creats amb l'aplicació Settle Up. S'ha calculat el temps de resolució per a cada cas 5 cops i s'ha fet la mitjana per cada cas. Analitzant les dades en un primer moment (figura \ref{fig:PLEM_1}) sembla que:

\begin{enumerate}
\item El temps d'execució és sempre molt baix
\item El temps depèn de forma logarítmica del nombre de nodes (N) del problema. ($N = 2*NC*ND + NC + ND + 2$)
\end{enumerate}

\begin{figure}[ht]
\centering
\includegraphics[scale=0.8]{PLEM_temps_2.png}
\caption{Temps per resoldre el PLEM en funció del nombre de nodes (N)}\label{fig:PLEM_1}
\end{figure}

Per a comprovar aquestes hipòtesis s'ha afegit un joc de dades inventat on el nombre de nodes fos més elevat. Amb aquest joc de dades el temps ja no era menor d'un segon (figura \ref{fig:PLEM_2}), com en els altres casos, sinó que estava prop dels 3 minuts. 

\begin{figure}[ht]
\centering
\includegraphics[scale=0.8]{PLEM_temps_1.png}
\caption{Temps per resoldre el PLEM en funció del nombre de nodes (N)}\label{fig:PLEM_2}
\end{figure}

Finalment i després de diverses proves s'ha comprovat que el temps en realitat depèn de forma logarítmica del nombre de deutors i creditors conjuntament, tal com es pot veure a la figura \ref{fig:PLEM_3}. S'han emprat 61 jocs de dades creats de manera que es calculessin la majoria de combinacions possibles de NC i ND amb un temps d'execució relativament baix. Com abans, cada cas s'ha calculat 5 cops i s'ha fet la mitjana dels 5 temps.

%TODO citar les dades

\begin{figure}[ht]
\centering
\includegraphics[scale=0.8]{PLEM_temps_3.png}
\caption{Temps per resoldre el PLEM en funció del nombre de creditors (NC) i dels deutors (ND)}\label{fig:PLEM_3}
\end{figure}

Aquest temps d'execució elevat és degut al \gls{BB}. Per trobar una solució amb valors reals, el temps és menor d'un segon, però al intentar trobar una solució amb valors enters tot assegurant què és òptima el temps ja es elevat.

\subsection{Algoritme final}
Finalment s'ha decidit calcular la solució òptima, restringint el programa a un temps d'execució de 3 segons, juntament amb una la resolució heurística que proposa David Vávra (2012, p. 6-7) \cite{Settle_up}. Si en aquest temps el programa troba una solució factible, és compararà amb la solució heurística per veure quin té millor valor de la funció objectiu. Si amb la resolució òptima no s'arriba a una solució factible s'utilitzarà l'aconseguida amb l'heurística. 

\section{Disseny de la base de dades}
A l'hora d'emmagatzemar les dades de la aplicació s'ha optat per fer servir \gls{SQLite} de forma local al \gls{smartphone} i fer ús del \gls{BAAS} Parse.com. Per tant s'ha dissenyat una base de dades local i una altre al núvol. La base de dades principal és la local i la del núvol està com a còpia de seguretat i per garantir la sincronització multidispositiu. Les taules que han de ser llegides per varies persones s'han partir en dues, una privada i una pública, al emmagatzemar-les al núvol de manera que el mínim d'informació és accessible per als usuaris que no l'han creat.

A la figura \ref{fig:db_sqlite} es pot veure el disseny de la base de dades local, a la figura \ref{fig:db_parse} es pot veure la base de dades al núvol i a la figura \ref{fig:db_private_shared} és pot veure quines columnes són privades i quines públiques a cada taula. 

\begin{figure}[ht]
\centering
\includegraphics[scale=0.44]{db_sqlite.png}
\caption{Base de dades local amb SQLite}\label{fig:db_sqlite}
\end{figure}

\begin{figure}[ht]
\centering
\includegraphics[scale=0.44]{db_parse.png}
\caption{Base de dades al núvol amb Parse.com}\label{fig:db_parse}
\end{figure}

\begin{figure}[ht]
\centering
\includegraphics[scale=0.8]{db_private_shared.png}
\caption{Columnes privades i compartides de cada taula}\label{fig:db_private_shared}
\end{figure}


\chapter{Impacte ambiental}
% Veure altres projectes: https://upcommons.upc.edu/pfc/?locale=es

\section{Impacte de l'estudi d'Experiència d'Usuari}
El projecte consisteix en un estudi sobre una aplicació per a \gls{smartphone}. El temps invertit en el desenvolupament d'aquest estudi es pot classificar en:

\begin{description}
\item[PFC] Inclou l'anàlisi, disseny i prototipatge (exceptuant els prototips programats), redacció de la memòria, entrevistes amb el tutor etc.
\item[Cursos] Són cursos sobre \gls{Android} i Java que s'han efectuat per poder aprendre a crear aplicacions per aquest sistema.
\item[Programació Android] Inclou el temps invertit programant els diversos prototips i parts d'aquests que han estat programats.
\end{description}


\begin{table}
\centering
\begin{tabular}{ | l | r | r |}
\hline
\headB{Part} & \headB{Temps invertit [h]} & \headB{Consum elèctric [kWh]} \\
\hline
PFC & 281 & 28,08\\
\hline
Cursos & 130 & 13,25\\
\hline
Programació & 471 & 48,00\\
\hline
\end{tabular}
\caption{Hores invertides en l'estudi i consum que representa}
\label{table:impact}
\end{table}

A la taula \ref{table:impact} i a la figura \ref{fig:impact} es poden veure les hores invertides en l'estudi. Durant tot l'estudi s'ha fet servir un ordinador amb una potència mitjana de 100W. A més a més, per a la programació i pels cursos s'ha fet servir un \gls{Nexus_5}, amb una potència mitjana de 2W. 


\begin{figure}[ht]
\centering
\includegraphics[scale=0.6]{Impact.png}
\caption{Temps invertit i consum associat}\label{fig:impact}
\end{figure}

Apart del consum energètic, també cal considerar el consum de paper. Tot i que s'ha intentat reduir al mínim el seu ús, en algunes parts de l'estudi era necessari, concretament s'ha emprat:

\begin{description}
\item[Plantilles per al disseny:] 22 
\item[Plantilles per a l'avaluació:] 25
\item[Varis:] 5
\end{description}

És a dir, en total s'ha emprat 52 fulls de paper.

\section{Impacte de l'ús de l'aplicació}
Tot i que en aquest projecte només s'ha arribat a crear un prototip, aquest ha estat dissenyat per a que en un futur es pugui posar al mercat una aplicació per a la gestió de despeses domèstiques. Com a tal, s'espera que en un futur modifiqui els hàbits d'alguns usuaris a l'hora de gestionar la seva economia personal, ajudant per una banda a prescindir de l'ús de paper. També s'espera que alguns usuaris canviïn l'ús d'ordinadors per al de \glspl{smartphone} amb el conseqüent estalvi energètic. Això però és més difícil de quantificar ja que depèn de la quantitat d'usuaris que acabin fent servir l'aplicació a més a més del sistema de gestió que utilitzaven prèviament. 


\chapter{Estudi de costos}
En primer lloc a la figura \ref{fig:gantt} es pot veure la planificació de les diverses activitats d'aquest projecte en forma de diagrama de Gantt. L'execució real s'ha adequat a la planificació.

\begin{figure}[ht]
\centering
\includegraphics[scale=0.8]{Gantt.png}
\caption{Diagrama de Gantt}\label{fig:gantt}
\end{figure}

De les activitats principals de la figura \ref{fig:gantt} s'extreuen els diversos costos de personal. Per a cada activitat s'han comptabilitzat les hores invertides i, utilitzant el cost per hora, s'ha calculat el cost total de cada activitat, com es pot veure a la taula \ref{table:cost}. 

A més del cost en recursos humans, també s'ha inclòs el cost dels aparells que s'han utilitzat. Per a calcular el cost unitari dels aparells s'ha comptabilitzat el temps que s'han fet servir respecte la seva vida útil. En quan a les llicencies, per a la de \textit{Microsoft Office} s'ha calculat de la mateixa manera que per als aparells. En canvi la llicència de \textit{Google Play} té un preu únic a pagar per a registrar-s'hi. 

Per al cost de l'energia emprada s'ha fet servir el consum elèctric calculat a l'apartat \ref{impacte}. Fent servir el cost mitjà del kWh a Espanya s'ha calculat el total referent a l'energia emprada.

Finalment el cost total del projecte és de 32.811 \euro { calculat} amb un 21\% d'IVA. El càlcul es pot veure al detall a la taula \ref{table:cost}


\begin{table}[htbp]
  \centering
    \begin{tabular}{| p{4cm} | r | p{3cm} | r | r | r |}
    \hline \headB{Concepte} & \headB{Valor [\euro ]} & \headB{Vida útil [mesos]} & \headB{Unitats} & \headB{Cost unitari} & \headB{Total [\euro ]} \\
    \hline \headA{Personal} & \headA{} & \headA{} & \headA{} & \headA{} & \headA{} \\ \hline
    Redacció PFC &       &       & 66h    & 40 \euro /h   & 2640,00 \euro \\
    Documentació estudis UX &       &       & 57h    & 40 \euro /h   & 2280,00 \euro \\
    Dissenys parts de l'aplicació &       &       & 36h    & 50 \euro /h   & 1800,00 \euro \\
    Estudi UX &       &       & 122h   & 50 \euro /h   & 6100,00 \euro \\
    Programació &       &       & 471h   & 30 \euro /h   & 14130,00 \euro \\
    
    \hline \headA{Aparells} & \headA{} & \headA{} & \headA{} & \headA{} & \headA{} \\ \hline
    Ordinador & 800   & 120   & 8 mesos    & 6,67 \euro /mes & 53,33 \euro \\
    Smartphone Nexus 5 & 350   & 36    & 8 mesos  & 9,72 \euro /mes & 77,78 \euro\\
    
    \hline \headA{Llicencies} & \headA{} & \headA{} & \headA{} & \headA{} & \headA{} \\ \hline
    Google play & 25 \euro   &       &       &       & 25,00 \euro \\
    Microsoft Office 2013 & 119   & 60    &       & 1,98  & 0,00 \euro \\
    
    \hline \headA{Energia} & \headA{} & \headA{} & \headA{} & \headA{} & \headA{} \\ \hline
    Electricitat &       &       & 89,33 kWh & 0,12 \euro /kWh & 10,72 \euro \\
	\hline \headA{Subtotal} & \headA{} & \headA{} & \headA{} & \headA{} & \headA{27116,83 \euro} \\ \hline    
    \hline \headB{Total (IVA 21\%)} & \headB{} & \headB{} & \headB{} & \headB{} & \headB{32811,37 \euro} \\ \hline
    \end{tabular}%
  \caption{Hores invertides, tarifa i cost associat}
\label{table:cost}
\end{table}%



\chapter{Conclusions}

\section{Problema del repartiment de despeses}
Al llarg d'aquest \ac{PFC} s'ha fet un estudi d'\ac{UX} enfocat a una aplicació per a la gestió de despeses personals. En quan als estudis \ac{UX} en si, s'ha deduït:

\begin{itemize}
\item L'anàlisi de com interaccionen els usuaris, sobretot a la primera iteració, és molt útil per entendre que volen els usuaris alhora que permet familiaritzar-se amb el producte/sistema a dissenyar.
\item En quan al disseny i prototipatge és important començar amb una fidelitat i anar incrementant-la a mesura que s'itera. D'aquesta manera s'estalvien molts recursos a les primeres iteracions, i és que no és útil detallar un disseny quan aquest és susceptible de ser canviat. %TODO Flinto al glossari
\item Amb les aplicacions com Flinto (www.flinto.com) es poden animar molt ràpidament dissenys per a crear prototips, però aquests no tindran la flexibilitat dels prototips de paper. En productes/sistemes simples, com l'estudiat, es creu que és suficient amb els prototips creats amb Flinto o semblants, però si el producte/sistema és complex seria millor emprar prototips de paper.
\item Tot i ser un procés iteratiu, si a la primera iteració s'obtenen molt bons resultats s'estalvien molts recursos al llarg de l'estudi. 
\item Pocs usuaris estan disposats a efectuar entrevistes o enquestes llargues. Quants menys productes (dels que ja existeixin) hagin d'avaluar, més usuaris estaran disposats a avaluar-los. 
\item Avaluar 10 aplicacions és pesat per als usuaris, si s'hagués treballat amb 4 o 5 hauria estat millor. Tot i això, al partir l'avaluació sumarial en 2 parts, fent que a la segona només s'avaluessin 3 aplicacions a facilitat obtenir dades de molts usuaris diferents. 
\item A l'avaluació sumarial, uns qüestionaris donen una bona idea de la qualitat obtinguda. Tot i això, si és possible per recursos, és recomanable complementar-ho obtenint mètriques \ac{UX}.
\item Tot i que existeixen moltíssimes aplicacions per gestionar les despeses personals, hi ha poques que garanteixin una bona \ac{UX}. I, evidentment, fer un estudi d'\ac{UX} garanteix que l'aplicació resultat agradi als usuaris. 
\item Un estudi d'\ac{UX} pot ser molt complex i requerir la participació d'un equip molt nombrós. És per això que és important adaptar-lo als recursos disponibles, tant econòmics, com personals o de temps. 
\end{itemize}

En quan a les aplicacions, tant en general com les que serveixen per gestionar despeses, s'ha arribat la conclusió que:

\begin{itemize}
\item Les parts que impliquen més feina al programar sovint són les que els usuaris els hi presten menys atenció, ja que és habitual que les consideren obvies. En aquest cas el repartiment de les despeses grupals ha estat una de les parts més complicades i pocs usuaris li han prestat atenció.
\item Hi ha moltes funcions que no emocionen positivament als usuaris, però que en cas que no funcionin correctament si què tenen un fort impacte negatiu en la seva \ac{UX}. La navegació per l'aplicació n'és un exemple, quan es pot navegar correctament l'usuari ni se n'adona, però en cas que no ho pugui fer es frustrarà.
\item L'apartat visual és molt important pels usuaris, si una aplicació no és maca, és complicat que agradi. Per tant és molt important que qualsevol prototip que vegin els usuaris tingui un bon acabat visual. 
\end{itemize}

I respecte a l'aplicació/prototip creat:

\begin{itemize}
\item S'ha aconseguit crear un prototip bastant semblant a com hauria de ser el producte final.
\item Si bé és cert que no s'ha pogut implementar totes les funcions que havia de tenir l'aplicació, si que comptava amb les funcions més importants, donant una bona idea de com serà l'aplicació si es publica en un futur. %TODO link objectius
\item L'aplicació compleix les funcions esmentades a l'apartat d'objectius encara que li manquin varies funcions que els usuaris volien.
\item El prototip final ha excedit les expectatives si es té en compte que només es creia factible fer proves de concepte.
\item El prototip té alguns errors importants d'\ac{UX} dels quals se'n té constància però que no s'han corregit per manca de temps o coneixements a l'hora de programar. %TODO llista coses a fer app
\end{itemize}

\section{Recomanacions}

\section{Futur de l'estudi/aplicació}
\chapter{Agraïments}

\begin{thebibliography}{9}

%Autor / Titol (Cursiva) / Dades de la publicació
\bibitem{Android_OS}
International Data Corporation. \textit{Smartphone OS Market Share, Q2 2014}. [\url{http://www.idc.com/prodserv/smartphone-os-market-share.jsp}, 20 d'Octubre del 2014].

\bibitem{UX_Book}
Rex Hartson. \textit{The UX Book: Process and Guidelines for Ensuring a Quality User Experience}. EEUU: Elseiver, 2012. %p. 19.

\bibitem{Cooper}
Alan Cooper. \textit{The Inmates Are Running the Asylum: Why High Tech Products Drive Us Crazy and How to Restore the Sanity}. EEUU: Sams Publishing, 2004.

\bibitem{Smashing_User_Experience_Design}
Smashing Magazine. \textit{User Experience Design}. Alemanya: Smashing Media GmbH, 2012. %p. 25-28.

\bibitem{user_centred_design}
Udacity. \textit{Personas and Use Cases Interview with Rich Fulcher}. [\url{https://www.youtube.com/watch?v=uL6xlI17gBU}, 22 de Novembre del 2014]. 

\bibitem{Contextual_Design}
Beyer i Holtzblatt. \textit{Contextual Design}. EEUU: Elseiver, 1998.

\bibitem{Nielsen_1993}
Jakob Nielsen. \textit{Usability Engineering}. Regne Unit: Academic Press, 1993.

\bibitem{Settle_up}
David Vávra. \textit{Mobile Application for Group Expenses and Its Deployment}. República txeca: Czech Technical University in Prague, 2012.

\bibitem{developing_android_apps}
Udacity. \textit{Developing Android Apps}. [\url{https://www.udacity.com/course/developing-android-apps--ud853}, 10 de Febrer del 2015]. 




\end{thebibliography}

\end{document}