\chapter{Introducció}

\section{Objectius del projecte}
L'objectiu d'aquest \ac{PFC} és fer un estudi de l'\ac{UX} per una aplicació que serveixi per a gestionar les despeses personals domèstiques. Amb aquest estudi es pretén arribar a definir com ha de ser aquesta aplicació per a que garanteixi una bona \ac{UX}. I un cop definida com ha de ser, se'n faran proves de concepte basades en aquesta definició. 

Concretament es busca dissenyar una aplicació que:
\begin{itemize}
\item Permeti enregistrar despeses i ingressos tot categoritzant-los.
\item Serveixi per a recordar els deutes (positius o negatius) que es tenen amb diverses persones.
\item Faciliti la gestió de despeses grupals alhora que permeti saldar els deutes minimitzant les transaccions entre els membres.
\item Permeti exportar/importar les dades, tant per a fer còpies de seguretat com per si l'usuari vol utilitzar-les externament.
\item Sigui intuïtiva i senzilla de fer servir.
\item Visualment sigui agradable i minimalista per a que sigui agradable i còmode per a l'usuari.
\end{itemize}

\section{Abast del projecte}
En aquest projecte s'estudiarà l'\ac{UX} per a l'aplicació esmentada tot definint quines utilitats i funcionalitats ha de tenir l'aplicació i com ha de ser la \ac{UI}. L'estudi s'enfocarà únicament en una aplicació per a dispositius mòbils que funcionen amb \gls{Android}, l'actual sistema operatiu més utilitzat\cite{Android_OS} en \glspl{smartphone} o telèfons intel·ligents. 
Finalment en quan al desenvolupament de l'aplicació, no es considera factible crear-la sencera amb totes els requeriments que es dedueixin amb l'estudi de l'\ac{UX}. És per això que es faran proves de concepte intentant apropar-se el màxim possible a com hauria de ser l'aplicació. 