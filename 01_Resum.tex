\chapter*{Resum} \label{sec:Resum}
\addcontentsline{toc}{chapter}{Resum}

En aquest projecte s'ha fet un estudi d'Experiència d'Usuari (UX) sobre les aplicacions de gestió de despeses personals per a \gls{Android}. Per a tal fi, en primer lloc s'ha definit què és i com es fan els estudis d'UX, explicant amb detall en que consisteixen les seves 4 etapes, que són: Anàlisi, Disseny, Implementació i Avaluació.

L'estudi en si ha començat analitzant com els usuaris gestionen les seves despeses i la seva satisfacció amb les aplicacions existents més rellevants per a \gls{Android}. Utilitzant aquest anàlisi s'ha definit els requeriments necessaris per a una aplicació per a la gestió de les despeses personals. Després d'aquest anàlisi previ s'ha fet un procés iteratiu on es dissenyava i s'implementaven prototips d'aquesta aplicació, augmentant-ne la fidelitat i el detall d'aquests a cada iteració. A continuació s'utilitzaven aquests prototips amb usuaris per a validar que satisfeien les seves necessitats, modificant els requeriments inicials si era necessari. 

Com a resultat d'aquest procés iteratiu s'ha creat un prototip programat que implementa les funcions més importants d'una aplicació per a gestionar les despeses domèstiques. Llavors s'ha comparat l'experiència d'usuari que proporcionava aquest prototip respecte les aplicacions ja existents al mercat. El resultat és un prototip que, proporciona la millor UX d'entre les aplicacions ja existents.

Per a poder dissenyar aquest prototip a calgut modelitzar el problema de la resolució dels deutes en grup, alhora que dissenyar i implementar un algoritme que ho solucioni. També s'ha comprovat el seu cost d'execució amb diversos jocs de dades. 

Finalment s'ha descrit els passos que caldria seguir per a poder passar del prototip programat a una aplicació completament funcional que es pugui penjar a les botigues d'aplicacions. 


